\chapter*{ВВЕДЕНИЕ}
\addcontentsline{toc}{chapter}{ВВЕДЕНИЕ}

Компьютерная графика играет важнейшую роль в современной жизни. 
Она позволяет нам видеть различные задачи в совершенно новом свете, а также создавать и исследовать визуально привлекательные и живые изображения.
С ее помощью мы можем представлять друг другу информацию с помощью цветного и анимированного содержимого. 
Получение качественного изображения необходимо во многих областях, включая медицину, искусство и игры.
Особенно важной становится задача генерации реалистичного изображения, с развитием технологий люди привыкают к все более
правдоподобной картинке. Чаще всего люди подмечают нереалистичность изображения при наблюдении света: его отражения и преломления \cite{compgraph_usage}.


Целью данной работы является разработка программного обеспечения, моделирующего отражения от геометрических тел.

Для достижения поставленной цели требуется решить следующие задачи:
\begin{enumerate}
	\item формализировать представление объектов сцены и описать их;
	\item проанализировать алгоритмы построения реалистичных изображений и теней;
	\item выбрать наилучшие алгоритмы для достижения цели  из рассмотренных;
	\item проанализировать полученную  модель взаимодействия света с объектами;
	\item выбрать програмные средства для реализации модели;
	\item реализовать полученную модель;
	\item создать интерфейс;
	\item провести замеры времени построения кадра от количества и типа примитивов на сцене.
\end{enumerate}
