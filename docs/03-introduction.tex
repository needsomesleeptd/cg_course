\chapter*{ВВЕДЕНИЕ}
\addcontentsline{toc}{chapter}{ВВЕДЕНИЕ}


Компьютерная графика --- совокупность методов и способов преобразования информации в графическое представление при помощи ЭВМ \cite{compgraph_importance}.
Конечном продуктом графики является изображение, представляющее собой визуальное представление данных, поданных для отображения \cite{Rodgers}. 

При необходимости получения фотореалистичных изображений, необходимо принимать во внимание условия освещения и оптические свойства материалов, задействованных в сцене. При расчете освещения данных изображений, необходимо учитывать отраженную  и преломленную составляющую света, для реалистичной визуализации наиболее важной составляющей является отраженная часть света \cite{compgraph_usage,real_images}. Таким образом, для получения реалистичного изображения, необходимо произвести расчет интенсивности отраженной составляющей света.


Целью данной работы является разработка и создание программного обеспечения, моделирующего отражения от геометрических тел.

Для достижения поставленной цели требуется решить следующие задачи:
\begin{enumerate}
	\item формализовать представление объектов сцены и описать их;
	\item проанализировать алгоритмы построения реалистичных изображений;
	\item выбрать оптимальные алгоритмы из рассмотренных для достижения цели;
	\item проанализировать полученную  модель взаимодействия света с объектами;
	\item выбрать программные средства для реализации модели;
	\item реализовать полученную модель и создать интерфейс;
	\item провести замеры времени построения кадра от количества и типа примитивов на сцене.
\end{enumerate}
