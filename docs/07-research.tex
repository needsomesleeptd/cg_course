\chapter{Исследовательская часть}


В данном разделе будет описано исследование зависимости среднего числа генерируемых кадров от числа и типа примитивов на сцене. Также будет описаны технические характеристики устройства, на котором проводились замеры и приведен анализ полученных результатов.

\section{Технические характеристики}

Технические характеристики устройства, на котором выполнялись замеры времени, представлены далее.

\begin{enumerate}
	\item Процессор	Intel(R) Core(TM) i7-9750H CPU @ 2.60GHz, 2592 МГц, ядер: 6, логических процессоров: 12.
	\item Оперативная память: 16 ГБайт.
	\item Операционная система: Майкрософт Windows 10 Pro \cite{windows}.
	\item Использованная подсистема: WSL2 \cite{WSL2}.
\end{enumerate}

При замерах времени ноутбук был включен в сеть электропитания и был нагружен только системными приложениями.

\section{Временные характеристики}
Результаты проведения временных замеров приведены в таблице~\ref{t:timings}.


\begin{table}[ht]
	\centering
	\caption{Зависимость среднего числа получаемых кадров в секунду от типов примитивов на изображении}
	\begin{tabular}{|c|c|c|c|c|}
		\hline
		n   & Сфера   & Куб     & Цилиндр & Конус   \\ \hline
		10  & 57.1419 & 54.9538 & 54.3946 & 55.4482 \\ \hline
		20  & 54.3279 & 50.9881 & 52.2229 & 54.4413 \\ \hline
		30  & 47.5699 & 41.1863 & 46.283  & 51.0728 \\ \hline
		40  & 42.3703 & 29.4745 & 37.9257 & 42.2258 \\ \hline
		50  & 33.1457 & 24.6811 & 31.8656 & 35.819  \\ \hline
		60  & 30.409  & 20.0419 & 27.4269 & 31.1626 \\ \hline
		70  & 26.0904 & 16.3896 & 24.1506 & 26.0774 \\ \hline
		80  & 23.2984 & 14.2909 & 21.2716 & 23.4664 \\ \hline
		90  & 22.4715 & 12.416  & 18.746  & 20.8784 \\ \hline
		100 & 18.6406 & 11.0113 & 16.4398 & 19.3409 \\ \hline
	\end{tabular}
	\label{t:timings}
\end{table}

В таблице~\ref{t:timings}~$n$ обозначает число видимых примитивов, иные столбцы обозначают тип примитива, для которого совершался замер. Шаг изменения числа примитивов равен 10, наблюдатель не изменял своей точки наблюдения между замерами, замеры производились при максимальном поколении луча равном 20, для замера времени был использован класс \texttt{QElapsedTimer}~\cite{QTimer}.
Замеры числа кадров  происходило 30 секунд, после чего результаты усреднялись и высчитывалось число кадров в секунду,  генерируемые примитивы были расставлены в виде решетки (см. \ref{img:primitives_positions}), со смещением 2 по осям X и Z. С помощью таблицы \ref{t:timings} был получен график на картинке \ref{img:course_cmp_time}.


\includeimage
{primitives_positions} % Имя файла без расширения (файл должен быть расположен в директории inc/img/)
{f} % Обтекание (без обтекания)
{H} % Положение рисунка (см. figure из пакета float)
{1\textwidth} % Ширина рисунка
{Пример расстановки примитивов для проведения замеров} % Подпись рисунка



\includeimage
{course_cmp_time} % Имя файла без расширения (файл должен быть расположен в директории inc/img/)
{f} % Обтекание (без обтекания)
{H} % Положение рисунка (см. figure из пакета float)
{1\textwidth} % Ширина рисунка
{Зависимость количестсва кадров в секунду от числа и типов примитивов на изображении} % Подпись рисунка

Из графика на рисунке \ref{img:course_cmp_time}, можно сделать вывод, что
наибольшее число кадров в секунду было получено при генерации конусов, наименьшее число кадров было получено при генерации кадров с наличием кубов. При увлечении числа примитивов на изображении число кадров убывает с ускорением --- это объясняется увлечением числа отражений, которые необходимо рассчитать при добавлении нового примитива. 

Куб имеет больший объем по сравнению с другими примитивами, ввиду этого большее число лучей попадают в данный объект, генерируя вторичные лучи для расчета, учитывая расстановку объектов, а также рассмотрение идеальных отражений, все отраженные лучи попадут в соседний куб, что увеличит время генерации кадра. 
Расчет пересечения с цилиндром и поиск его нормали требует отдельной проверки пересечения луча с его основаниями, этим объясняется большее время получения кадра, чем время получения кадра у сферы и конуса. При 100 объектах на сцене при наблюдении сфер в среднем было получено 18.64 кадра в секунду, при рассмотрении конусов было получено в 1.04 раза больше кадров, сцена с кубами была сгенерирована  в 1.70 меньше раз, при генерации кадров с цилиндрами было получено  в 1.13 меньше кадров, чем при использовании сфер.








