\chapter*{\hfill{\centering  ЗАКЛЮЧЕНИЕ}\hfill}
\addcontentsline{toc}{chapter}{ЗАКЛЮЧЕНИЕ}

В результате исследования, было определено, что время генерации кадра зависит от типов объектов на сцене. Наибольшее число кадров было получено при рассмотрении сцены с наличием сфер, что объясняется малыми размерами сферы и малым числом математических операций, необходимых для поиска нормали и точки пересечения со сферой по сравнению с конусом и цилиндром. Наименьшее число кадров было получено
при использовании кубов, что объясняется большим числом отраженных от них лучей.При 100 объектах на сцене наибольшее число кадров за 30 секунд было получено при наблюдении сфер (в среднем было получено 18.64 кадра в секунду), при рассмотрении конусов было получено в 1.04 раза больше кадров, сцена с кубами была сгенерирована  в 1.70 меньше раз, при генерации кадров с цилиндрами было получено  в 1.13 меньше кадров, чем при использовании сфер.

Поставленная цель: разработка и создание программного обеспечения, моделирующего отражения от геометрических тел.
Для поставленной цели были выполнены все задачи:
\begin{enumerate}
	\item формализировать представление объектов сцены и описать их;
	\item проанализировать алгоритмы построения реалистичных изображений и теней;
	\item выбрать наилучшие алгоритмы для достижения цели  из рассмотренных;
	\item проанализировать полученную  модель взаимодействия света с объектами;
	\item выбрать програмные средства для реализации модели;
	\item реализовать полученную модель и создать интерфейс;
	\item провести замеры времени построения кадра от количества и типа примитивов на сцене.
\end{enumerate}