\chapter*{\hfill{\centering  ЗАКЛЮЧЕНИЕ}\hfill}
\addcontentsline{toc}{chapter}{ЗАКЛЮЧЕНИЕ}

В результате исследования, было определено, что время генерации кадра зависит от типа объекта на сцене. Наибольшее число кадров при 100 объектов на сцене было получено при рассмотрении сцены с наличием конусов, что объясняется малыми размерами конуса, при рассмотрении сфер результат был близок к результату конуса, ввиду того, что для построения нормали к сферы необходима одна операция вычитания векторов, для получения точки пересечения необходимо решить уравнение, с введением только одной дополнительной переменной, без необходимости дополнительных математических преобразований. Наименьшее число кадров было получено
при использовании кубов, что объясняется большим числом отраженных от них лучей и их наибольшим объемом по сравнению с иными примитивами. При 100 объектах на сцене при наблюдении сфер в среднем было получено 18.64 кадра в секунду, при рассмотрении конусов было получено в 1.04 раза больше кадров, сцена с кубами была сгенерирована  в 1.70 меньше раз, при генерации кадров с цилиндрами было получено  в 1.13 меньше кадров, чем при использовании сфер.

Поставленная цель: разработка и создание программного обеспечения, моделирующего отражения от геометрических тел был выполнена.

Для поставленной цели были выполнены все задачи:
\begin{enumerate}
	\item формализированы и описаны представления объектов сцены;
	\item проанализировать алгоритмы построения реалистичных изображений и теней;
	\item выбрать наилучшие алгоритмы для достижения цели  из рассмотренных;
	\item проанализировать полученную  модель взаимодействия света с объектами;
	\item выбрать програмные средства для реализации модели;
	\item реализовать полученную модель и создать интерфейс;
	\item провести замеры времени построения кадра от количества и типа примитивов на сцене.
\end{enumerate}