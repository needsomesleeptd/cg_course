\chapter*{Заключение}
\addcontentsline{toc}{chapter}{Заключение}  
В процессе выполнения задачи было рассмотрено несколько алгоритмов создания реалистичных изображений и моделей освещения. Для решения поставленной
задачи были выбраны наилучшие из них.

Был подробно рассмотрен алгоритм трассировки лучей, а также глобальная модель освещения, для создания реалистичных отражений. Также был рассмотрен
аналитический поиск точки пересечения луча с примитивами. Таким образом было спроектировано программное  обеспечение, позволяющее располагать фигуры на сцене и
наблюдать их  реалистичные отражения.

С результатами данной работы можно приступать к созданию программного обеспечения и проанализировать зависимость скорости отрисовки сцены от 
количества отражения лучей и количества примитивов на сцене. Также полученное ПО можно легко менять, совершенствуя и изменяя рассмотренные алгоритмы.