\chapter{Введение}

Компьютерная графика играет важнейшую роль в современной жизни. 
Она позволяет нам видеть различные задачи в совершенно новом свете, а также создавать и исследовать визуально привлекательные и живые изображения.
С ее помощью мы можем представлять друг другу информацию с помощью цветного и анимированного содержимого. 
Получение качественного изображения необходимо во многих областях, включая медицину, искусство и игры.
Особенно важной становится задача генерации реалистичного изображения, с развитием технологий люди привыкают к все более
правдоподобной картинке. Чаще всего люди подмечают нереалистичность изображения при наблюдении света: его отражения и преломления.


Целью данной работы является разработка программного обеспечения, моделирующего отражения от геометрических тел.

Для достижения поставленной цели требуется решить следующие задачи:
\begin{enumerate}
	\item Формализировать представление объектов сцены и описать их.
	\item Проанализировать алгоритмы построения реалистичных изображений и теней.
	\item Выбрать наилучшие алгоритмы для достижения цели  из рассмотренных.
	\item Проанализировать полученную  модель взаимодействия света с объектами.
	\item Выбрать програмные средства для реализации модели.
	\item Реализовать полученную модель.
	\item Создать интерфейс.
	\item Провести замеры временных характеристик полученной модели.
\end{enumerate}
