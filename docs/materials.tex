
\begin{thebibliography}{9}
	\bibitem{Rodgers}
	Роджерс Д. Алгоритмические основы машинной графики. - 1-е изд. - Москва: Мир, 1989. - 512 с.
	\bibitem{global_model}
	Модель глобального освещения с трассировкой лучей [Электронный ресурс]. – Режим доступа: https://vunivere.ru/work71759/page3 (дата обращения 15.07.23)
	\bibitem{modern_ray_tracing}
	Современное состояние методов расчета глобальной освещенности в задачах реалистичной компьютерной графики [Электронный ресурс]. – Режим доступа: https://cyberleninka.ru/article/n/sovremennoe-sostoyanie-metodov-raschyota-globalnoy-osveschyonnosti-v-zadachah-realistichnoy-kompyuternoy-grafiki/viewer (дата обращения 15.07.23)
	\bibitem{SSR}
	Screen Space Reflection Techniques [Электронный ресурс]. – Режим доступа: https://ourspace.uregina.ca/handle/10294/9245 (дата обращения 15.07.23)
	\bibitem{reflexion_types}
	Отражение в играх. Как работают, различия и развитие технологий [Электронный ресурс]. – Режим доступа: https://clck.ru/34zZCf (дата обращения 15.07.23)
	\bibitem{simple_reflexion_types}
	Простые модели освещения  [Электронный ресурс]. – Режим доступа: https://grafika.me/node/344 (дата обращения 15.07.23)
	\bibitem{primitives_raytracing_equations}
	Ray tracing primitives  [Электронный ресурс]. – Режим доступа: https://www.cl.cam.ac.uk/teaching/1999/AGraphHCI/SMAG/node2.html (дата обращения 20.07.23)
	\bibitem{perspective_raytracing}
	Ray tracing  [Электронный ресурс]. – Режим доступа: https://www.mauriciopoppe.com/notes/computer-graphics/ray-tracing/ (дата обращения 23.07.23)


\end{thebibliography}

\addcontentsline{toc}{chapter}{Список литературы}