\documentclass[a4paper,14pt,unknownkeysallowed]{extreport}

\usepackage{cmap} % Улучшенный поиск русских слов в полученном pdf-файле
\usepackage[T2A]{fontenc} % Поддержка русских букв
\usepackage[utf8x]{inputenc} % Кодировка utf8
\usepackage[english,russian]{babel} % Языки: русский, английский
\usepackage{enumitem}


\usepackage{threeparttable}

\usepackage[14pt]{extsizes}

\usepackage{caption}
\captionsetup{labelsep=endash}  
\captionsetup[figure]{name={Рисунок}}

% \usepackage{ctable}
% \captionsetup[table]{justification=raggedleft,singlelinecheck=off}

\usepackage{amsmath}

\usepackage{geometry}

\usepackage{svg}

\geometry{left=30mm}
\geometry{right=10mm}
\geometry{top=20mm}
\geometry{bottom=20mm}

\usepackage{titlesec}
\titleformat{\section}
	{\normalsize\bfseries}
	{\thesection}
	{1em}{}
\titlespacing*{\chapter}{0pt}{-30pt}{8pt}
\titlespacing*{\section}{\parindent}{*4}{*4}
\titlespacing*{\subsection}{\parindent}{*4}{*4}

\usepackage{setspace}
\onehalfspacing% Полуторный интервал

\frenchspacing
\usepackage{indentfirst} % Красная строка

\usepackage{titlesec}
\titleformat{\chapter}{\LARGE\bfseries}{\thechapter}{20pt}{\LARGE\bfseries}
\titleformat{\section}{\Large\bfseries}{\thesection}{20pt}{\Large\bfseries}

\usepackage{multirow}
\usepackage{listings}
\usepackage{xcolor}

% Для листинга кода:
\lstset{%
	language=C++,   					% выбор языка для подсветки	
	basicstyle=\small\sffamily,			% размер и начертание шрифта для подсветки кода
	numbers=left,						% где поставить нумерацию строк (слева\справа)
	numberstyle=\tiny,		     		% размер шрифта для номеров строк
	stepnumber=1,						% размер шага между двумя номерами строк
	numbersep=5pt,						% как далеко отстоят номера строк от подсвечиваемого кода
	frame=single,						% рисовать рамку вокруг кода
	tabsize=4,							% размер табуляции по умолчанию равен 4 пробелам
	captionpos=t,						% позиция заголовка вверху [t] или внизу [b]
	breaklines=true,					
	breakatwhitespace=true,				% переносить строки только если есть пробел
	backgroundcolor=\color{white},
	basicstyle=\footnotesize\ttfamily,
	keywordstyle=\color{blue},
	stringstyle=\color{red},
	commentstyle=\color{gray}
	showspaces=false,
    showstringspaces=false
}



\usepackage{pgfplots}
\usetikzlibrary{datavisualization}
\usetikzlibrary{datavisualization.formats.functions}




\usepackage{graphicx}
\graphicspath{ {images/} }
\newcommand{\img}[3] {
	\begin{figure}[h!]
		\center{\includegraphics[height=#1]{img/#2}}
		\caption{#3}
		\label{img:#2}
	\end{figure}
}


\usepackage[justification=centering]{caption} % Настройка подписей float объектов

\usepackage[unicode,pdftex]{hyperref} % Ссылки в pdf
\hypersetup{hidelinks}

\usepackage{csvsimple}

\newcommand{\code}[1]{\texttt{#1}}

\usepackage{longtable}

\usepackage{array}
\usepackage{booktabs}
\usepackage{floatrow}
\usepackage{ulem}


\floatsetup[longtable]{LTcapwidth=table}

\def\UrlBreaks{\do\/\do-\do\_}

\makeatletter
\renewcommand*\l@chapter[2]{%
  \ifnum \c@tocdepth >\m@ne
    \addpenalty{-\@highpenalty}%
    \vskip 1.0em \@plus\p@
    \setlength\@tempdima{1.5em}%
    \begingroup
      \parindent \z@ \rightskip \@pnumwidth
      \parfillskip -\@pnumwidth
      \leavevmode \bfseries
      \advance\leftskip\@tempdima
      \hskip -\leftskip
      #1\nobreak\normalfont\leaders\hbox{$\m@th
        \mkern \@dotsep mu\hbox{.}\mkern \@dotsep
        mu$}\hfill\nobreak\hb@xt@\@pnumwidth{\hss #2}\par
      \penalty\@highpenalty
    \endgroup
  \fi}
\makeatother

%packages for  title
\renewcommand{\underset}[2]{\ensuremath{\mathop{\mbox{#2}}\limits_{\mbox{\scriptsize #1}}}} % Use \underset in normal environment, not only in math

\usepackage{xparse} % \NewDocumentCommand for creating custom commands

% Write stuff under underlined text
\NewDocumentCommand{\ulinetext}{O{3cm} O{c} m m} % O - optional; m - mandatory
{\underset{#3}{\uline{\makebox[#1][#2]{#4}}}}

% Test bold underline
\NewDocumentCommand{\bolduline}{}{\bgroup\markoverwith
{\rule[-0.5ex]{2pt}{2pt}}\ULon}

% Display numberless sections in toc
\NewDocumentCommand{\snumberless}{m}
{\section*{#1}\addcontentsline{toc}{section}{#1}}

\usepackage{tabularx}

% Make sections also use dotted lines; normal font instead of bold in toc
\usepackage{tocloft}
\renewcommand{\cftsecdotsep}{\cftdotsep}
\renewcommand{\cftsecfont}{\normalfont}
\renewcommand{\cftsecpagefont}{\normalfont}

\usepackage{xr}