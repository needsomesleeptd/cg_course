\chapter{Технологическая часть}
\section{Выбор и обоснование языка программирования и среды разработки}
При выборе языка программирования важно учитывать много факторов в 
зависимости от задачи. Из-за поставленной задачи на сцене будет несколько объектов, которые будут
перемещаться и изменяться, что будет создавать большую нагрузку на процессор.
В качестве языка программирования (ЯП) был выбран Си++. На это есть 
несколько причин:
\begin{enumerate}
    \item Данный язык программирования имеет большую производительность, что играет большую роль в выбранном алгоритме.
    \item Данная задача просто реализуется при использовании объектно-ориентированного подхода, который поддерживается данным языком.
    \item Используя стандартную библиотеку языка возможно создание всех требуемых структур данных, выбранных в результате проектирования.
\end{enumerate}
В качестве среды разработки был выбран CLion, так как:
\begin{enumerate}
    \item Использует утилиту CMake, что позволит конфигурировать проект независимо от платформы.
    \item Упрощает использование библиотек с помощью удобного интерфейса.
    \item Обладает необходимым функционалом для сборки, профилирования и отладки программ.
\end{enumerate}

\section{Реализация алгоритмов}
\begin{lstlisting}
    
\end{lstlisting}